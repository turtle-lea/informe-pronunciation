\chapter*{Resumen}

Los avances tecnol\'{o}gicos de las \'{u}ltimas d\'{e}cadas han
posibilitado el desarrollo de sistemas
autom\'{a}ticos de Asistencia Computarizada para el Aprendizaje de Idiomas (ACAI).
Estos sistemas brindan ayuda a estudiantes de segundos idiomas
en diversos campos, entre las cuales se destacan la gram\'{a}tica, el vocabulario
y la pronunciaci\'{o}n. En el presente trabajo nos concentramos en una forma
particular de asistencia relacionada con el \'{u}ltimo campo:
\textcolor{red}{
la evaluaci\'{o}n
de la pronunciaci\'{o}n, que consiste en decidir si los segmentos de habla presentes
en una determinada
grabaci\'{o}n fueron pronunciados de forma correcta o incorrecta. Dicho an\'{a}lisis
puede realizarse a distintos niveles tales como oraci\'{o}n, palabra o fono.
}
%Este proceso consiste en la generación de un valor
%a partir una frase grabada por un estudiante, que mida cuán bien
%fue pronunciada esa frase,
%o cada palabra o fono dentro de esa frase.

Actualmente, las estimaciones m\'{a}s confiables de la evaluaci\'{o}n de la pronunciaci\'{o}n
son obtenidas a nivel de p\'{a}rrafo u oraciones largas, disminuyendo la precisi\'{o}n
de los sistemas a medida que se reduce la duraci\'{o}n
(y por lo tanto la cantidad de informaci\'{o}n) del segmento de habla a considerar.
Sin embargo, los sistemas de evaluaci\'{o}n de la pronunciaci\'{o}n
que trabajan con unidades de habla
m\'{a}s cortas, como por ejemplo el fono,
permiten poner el foco en errores espec\'{i}ficos del estudiante y
pueden ser utilizados por ni\~nos a\'{u}n incapaces de pronunciar frases
demasiado largas. Por esta raz\'{o}n,
en este trabajo nos concentramos en m\'{e}todos de evaluaci\'{o}n de la pronunciaci\'{o}n
a nivel fono.

Los m\'{e}todos tradicionalmente utilizados para evaluar
la pronunciaci\'{o}n a nivel fono
est\'{a}n basados en m\'{e}todos generativos a partir de
modelos de mezclas Gaussianas (GMMs). Generalmente,
para cada fono se
entrena un GMM por clase (pronunciaci\'{o}n correcta e incorrecta),
aplicando luego t\'{e}cnicas tales como el Cociente de
Verosimilitud (\textit{Likelihood-Ratio} en ingl\'{e}s) entre ambos modelos realizar
la evaluaci\'{o}n. En un trabajo anterior en
el \'{a}rea de evaluaci\'{o}n de la pronunciaci\'{o}n a nivel fono,
se explor\'{o}
un m\'{e}todo discriminativo basado en M\'{a}quinas de Vectores de Soporte (SVM) entrenado
con atributos llamados \textit{supervectores}, que produce resultados
ligeramente mejores a los m\'{e}todos generativos com\'{u}nmente utilizados en el campo.
Los \textit{supervectores} para cada fono se obtienen
a partir de un proceso de adaptaci\'{o}n de un
GMM global entrenado con la totalidad
de las muestras de dicho fono.

En el presente trabajo, tomamos como
base y punto de referencia el modelo SVM entrenado con
supervectores para explorar nuevos atributos en el \'{a}rea de evaluaci\'{o}n de la
pronunciaci\'{o}n a nivel fono.
Si bien tanto GMMs como
supervectores modelan las caracter\'{i}sticas ac\'{u}sticas de
bajo nivel del segmento
de habla a considerar,
no tienen en cuenta el comportamiento temporal de las mismas.
% contienen información de los atributos acústicos del habla a lo
% largo de toda la pronunciación del fono, no tienen en cuenta el orden en el que se
% producen.
Por este motivo, en esta ocasi\'{o}n
estudiamos atributos din\'{a}micos
que modelan de manera directa el comportamiento temporal de dichas
caracter\'{i}sticas ac\'{u}sticas.
Para ello, cada una es aproximada
de manera independiente por una funci\'{o}n,
a partir de la cual se extraen los atributos din\'{a}micos.
% Para ello, cada
% característica es modelada de forma independiente y aproximada por una función.
% modelen las dependencias temporales presentes en cada muestra.
% En este trabajo nos basamos en modelos SVM para
% En particular, estudiamos atributos dinámicos que modelan las dependencias
% temporales de cada muestra.
% Qué modelamos? Evolución de los atributos acústicos del habla a lo largo de la pronunciación del fono
% Por qué? Los supervectors están computados en base a los valores de los atributos
% acústicos del fono. No tienen en cuenta el orden en el eque dichos valores son observados
% Cómo? Cada atributo es modelado de forma independiente y aproximado por una función
% Parametrization techniques
Dos t\'{e}cnicas de aproximaci\'{o}n son evaluadas como posibles alternativas:
Polinomios de Legendre y Transformada Discreta del Coseno (DCT).
El objetivo
es analizar si
los atributos din\'{a}micos propuestos tienen informaci\'{o}n
complementaria a la provista por los supervectores.

Entrenamos y evaluamos los m\'{e}todos base y los propuestos usando una base de datos no nativa
de Espa\~nol Latino, correspondiente a 206 hablantes estadounidenses, estudiantes de Espa\~nol.
% Las grabaciones corresponden a 206 hablantes estadounidenses pronunciando distintas
% frases en español latino.
La base de datos est\'{a} conformada por 2550 grabaciones alcanzando
un total de 130.000 instancias de fonos etiquetadas
por transcriptores profesionales.
Los resultados muestran que para un subconjunto de fonos, la combinaci\'{o}n de supervectores
con los atributos din\'{a}micos efectivamente reduce los errores durante la clasificaci\'{o}n,
soportando la
hip\'{o}tesis de que ambos tipos de atributos contienen informaci\'{o}n complementaria.

\bigskip

\noindent\textbf{Palabras claves:} Asistencia Computarizada para Aprendizaje de Idiomas, Evaluaci\'{o}n de la Pronunciaci\'{o}n, Fono, M\'{a}quinas de Vectores de Soporte, Modelo de Mezclas Gaussianas, Supervectores, Polinomios de Legendre, Transformada Discreta del Coseno

