\chapter*{Resumen}

Los avances tecnológicos de las \textcolor{red}{últimas décadas} han
posibilitado el desarrollo de sistemas
automáticos de Asistencia Computarizada para el Aprendizaje de Idiomas (ACAI).
\textcolor{red}{Estos sistemas brindan ayuda a estudiantes de segundos idiomas
en diversos campos, entre las cuales se destacan la gramática, el vocabulario
y la pronunciación. En el presente trabajo nos concentramos en una forma
particular de asistencia relacionada con el último campo: la puntuación
de la pronunciación. Este proceso consiste en la generación de un valor
a partir una frase grabada por un estudiante, que mida cuán bien
fue pronunciada esa frase,
o cada palabra o fonema dentro de esa frase.}

Actualmente, las estimaciones más confiables de la puntuación de la pronunciación
son obtenidas a nivel de párrafo u oraciones largas, disminuyendo la precisión
de los sistemas a medida que se reduce la duración
(y por lo tanto la cantidad de información) del segmento de habla a considerar.
Sin embargo, los sistemas ACAI que trabajan con unidades de habla
más cortas, como por ejemplo el fono,
permiten poner el foco en errores específicos del estudiante y
pueden ser utilizados por niños aún incapaces de pronunciar frases
demasiado largas. \textcolor{red}{Por esta razón,}
en este trabajo nos concentramos en métodos de puntuación de la pronunciación
que califican a nivel fono.

\textcolor{red}{Los métodos tradicionalmente utilizados para calcular
puntajes de pronunciación a nivel fono
están basados en métodos generativos a partir de
modelos de mezclas Gaussianas (GMMs). Generalmente,
para cada fono se
entrena un GMM por clase (pronunciación correcta e incorrecta),
aplicando luego técnicas tales como el Cociente de
Verosimilitud (\textit{Likelihood-Ratio} en inglés) entre ambos modelos para
computar el puntaje.} En un trabajo anterior en
el área de calificación de pronunciación a nivel fono,
se exploró
un método discriminativo basado en Máquinas de Vectores de Soporte (SVM) entrenado
con atributos llamados \textit{supervectores}, que produce resultados
ligeramente mejores a los métodos generativos comúnmente utilizados en el campo.
Los \textit{supervectores} para cada fono se obtienen
a partir de un proceso de adaptación de un
GMM global entrenado con la totalidad
de las muestras de dicho fono. \textcolor{red}{
Tanto el método generativo basado en GMMs,
como el discriminativo basado en SVMs y entrenado con \textit{supervectores}
modelan características acústicas
de bajo nivel sin considerar sus comportamientos temporales.}
% que produce resultados ligeramente mejores a los métodos
% generativos comúnmente utilizados en este campo.
% Los \textit{supervectores} se obtienen a partir de
% un proceso de adaptación de un GMM global entrenado a partir de la totalidad
% de las muestras de cada fono.

En el presente trabajo, tomamos como
base y punto de referencia el modelo SVM entrenado con
supervectores para explorar nuevos atributos en el área de puntuación de la
pronunciación a nivel fono.
Si bien los supervectores contienen información de los atributos acústicos del habla a lo
largo de toda la pronunciación del fono, no tienen en cuenta el orden en el que se
producen. \textcolor{red}{Por este motivo, en esta ocasión
estudiamos atributos dinámicos
que modelen \textcolor{red}{de manera directa el comportamiento temporal de las
características de corto plazo}.
Para ello, cada característica es aproximada
de manera independiente por una función,
a partir de la cual se extraen los atributos dinámicos.}
% Para ello, cada
% característica es modelada de forma independiente y aproximada por una función.
% modelen las dependencias temporales presentes en cada muestra.
% En este trabajo nos basamos en modelos SVM para
% En particular, estudiamos atributos dinámicos que modelan las dependencias
% temporales de cada muestra.
% Qué modelamos? Evolución de los atributos acústicos del habla a lo largo de la pronunciación del fono
% Por qué? Los supervectors están computados en base a los valores de los atributos
% acústicos del fono. No tienen en cuenta el orden en el eque dichos valores son observados
% Cómo? Cada atributo es modelado de forma independiente y aproximado por una función
% Parametrization techniques
Dos técnicas de aproximación son evaluadas como posibles alternativas:
Polinomios de Legendre y Transformada Discreta del Coseno (DCT).
El objetivo
es analizar si
\textcolor{red}{los atributos dinámicos propuestos tienen información
complementaria a la provista por los supervectores}.
% tanto los supervectores como atributos dinámicos
% contienen información complementaria,
% a partir de la cual pueda mejorarse la efectividad de modelos futuros.

% In this work
% Dynamic Features $\rightarrow$
% Temporal information $\rightarrow$
% DCT and Legendre Polynomials $\rightarrow$
% score combination - features combination $\rightarrow$
% phone dependent $\rightarrow$.
% Phonetically transcribed Spanish database of around...

Entrenamos y evaluamos los métodos base y los propuestos usando una base de datos no nativa
de Español Latino, correspondiente a 206 hablantes estadounidenses, estudiantes de Español.
% Las grabaciones corresponden a 206 hablantes estadounidenses pronunciando distintas
% frases en español latino.
La base de datos está conformada por 2550 grabaciones alcanzando
un total de 130.000 instancias de fonos etiquetadas
por transcriptores profesionales.
Los resultados muestran que para un subconjunto de fonos, la combinación de supervectores
con los atributos dinámicos efectivamente reduce los errores durante la clasificación,
soportando la
hipótesis de que ambos tipos de atributos contienen información complementaria.

\bigskip

\noindent\textbf{Palabras claves:} Asistencia Computarizada para Aprendizaje de Idiomas, Puntuación de la Pronunciación, Fono, Máquinas de Vectores de Soporte, Modelo de Mezclas Gaussianas, Supervectores, Polinomios de Legendre, Transformada Discreta del Coseno, Combinación de Atributos, Combinación de Puntajes

