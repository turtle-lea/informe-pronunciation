\textit{K-Fold Cross Validation} \cite{svm_jwht}
is a method which aims to improve the accuracy in the
estimate of the errors for a given model by fitting and testing the model using different
subsets of the available development data. The error estimate is more robust than
that computed when fitting the model only once using all the available training data.

In order to keep the error estimates unbiased
the set of speakers in each of the folds must be disjoint (the same as
when splitting the dataset into development and hold-out).
The partitioning of the development set
is then carried out at speaker level, and each fold is obtained by gathering all the
utterances pronounced by the speakers in that fold.

To apply K-Fold Cross Valdation,
the development set is divided in $k$ groups or folds of approximately equal size.
Then, for each fold $i \ (1 \leq i \leq k)$ the following procedure is repeated:

\begin{enumerate}
  \item A new model is fitted using the instances of all the folds but the i\textsuperscript{th} fold.
  \item The fitted model is tested against the instances of the i\textsuperscript{th} fold.
\end{enumerate}

The results of all the iterations are then combined to obtain a final estimate of the error.

K-Fold Cross Validation maximizes the number of both training and testing instances.
Every instance is used exactly once for testing, and $k-1$ times for training the different
models. It also allows an unbiased estimation of the results,
because in none of the iterations the same instance is used for both training and testing the
model.

In the current work, Cross Validation is performed in the development set using 4 folds.
On each iteration, the different models and features are computed using the instances of
three of the folds and tested on the other one.
EER is computed by gathering the scores from all the folds.
This approach for EER computation leads to a more realistic statistic than computing an EER
per fold and taking the average, because the latter would lead to optimistic results.
