The dataset for the current work is obtained from a Latin-American Spanish speech database
\cite{database_collection}.  The same database has been used in the previous work of the
current line of \mbox{investigation \cite{main}}.

Even though the database includes recordings by natives and nonnative
speakers, only the nonnative recordings were used in the current work.
The nonnative data consists in 2550 utterances of read speech of different sentences
taken from Spanish newspaper data, adding up to
a total of 130,000 phone instances. The utterances were pronounced by 206 native American English
speakers who had studied some Spanish locally or abroad. Their levels of proficiency were varied,
and an attempt was made to balance the number of speakers by level of proficiency as well as by
gender.

The set of sentences was chosen to maximize the number of
occurrences of potential pronunciation problems gathered by a linguist and a Spanish language
instructor. It was also intended to include phones in different contexts that are known to be
difficult for native American English speakers to pronounce.
Examples are the dipthtong [eu], /r/ after [l] [n] and [s] (which should be
thrilled), and [p], [t] and [k] in any context (which nonnatives may aspirate).

Four native Spanish-speaking phoneticians provided the detailed phonetic transcriptions
for the nonnative utterances. Labels
for each phone instance were then generated by comparing the phonetician's transcriptions with
the canonical transcriptions of the sentence: transcribed phone instances that
matched the cannonical transcription were labeled as correct, while phone instances that did not
match the cannonical transcription were labeled as mispronounced. When performing assessment
tasks that involve several raters, usually there is a certain level of disagreement among the
transcribers. In fact, not all the phones were transcribed with the same level of reliability.
\textit{Cohen's Kappa} coefficient $K$ \cite{kappa} is used to describe how reliably the
transcribers agree on the transcriptions for each of the native phones. For eight of the phones
($/\beta/$, $/\delta/$, $/\gamma/$, /b/, /w/, /m/, /i/, /s/), all four transcribers showed at least
a moderate level of agreement (using $K > 0.4$ to mean ``moderate'' agreement).

% Phonetic alignments were generated using the EduSpeak \cite{edu_speak} HMM-based recognizer.
% Phonetic alignments are used to define the temporal boundaries of the instances present in the
% utterances, when obtaining th e MFCCs for those phone instances.
