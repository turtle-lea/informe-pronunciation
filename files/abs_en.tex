\chapter*{Abstract}
\addcontentsline{toc}{chapter}{\numberline{}Abstract}

\noindent
Computer-Assisted language learning (CALL) systems have been
evolving for the last few years due to
technological advances. An important subset of these tools are those which focus in
automatic pronunciation scoring and pronunciation assesment.
\textcolor{red}{
  Currently, whenever performing pronunciation assessment, the most reliable estimates are
  obtained from paragraphs or long sentences. On the other hand, the smaller the unit
  (and therefore the smaller the amount of speech data), the less precise is
  the estimate of the assessment.
  However, CALL systems that operate at shorter levels, such
  as phone, not only can point out specific errors produced by the students but also
  can be used by children that still have difficulties in pronouncing long sentences.
  In the current work we will focus on phone-level pronunciation scoring methods.
}
% Particularly, we
% are interested in systems that provide phone-level feedback on pronunciation quality,
% because they not only can point out specific errors to the students but they also
% can be used by children that are still unable to pronounce whole sentences or paragraphs.

In a previous work in the pronunciation assessment field at phone level, a discriminative
approach based on \textit{Support Vector Machines} (SVM) trained on special features
called \textit{supervectors} was explored, leading to slightly better results than
generative models such as \textit{Gaussian Mixture Models} (GMMs), which are usually
used in the pronunciation assessment field. Supervectors are derived from
adapted GMMs that are trained using all the available instances for a given phone.

\textcolor{red}{
In the current work, we use as reference and baseline the SVM model
trained on supervectors in order to explore new features. Even though supervectors
summarize the acoustic information extracted along the whole utterance,
they don't provide information about the order on which the acoustic features are
produced. Because of that reason, we study features that model the dynamics of the
acoustic features over time. In order to do so, each feature is modeled independently
by a parametric function.}
% In this work we use SVM models to explore dynamic features that model
% temporal dependencies in each instance.
Two different parameterization techiniques are studied:
\textit{Legendre Polynomials} and \textit{Discrete Cosine Transform} (DCT).
The objective is to analyse if both supervectors and dynamic features carry
complementary information, which can be used to improve future models. In order to do
so, two alternatives are proposed: A Features Combination, where a single classifier
is trained on the combination of both supervectors and dynamic features, and a
Score Combination, which is obtained by combining the scores of two SVM classifiers
trained independently on each set of features.

\textcolor{red}{We train and test the baseline and the proposed methods on a}
% The dataset for the current work is obtained from a Latin-American Spanish speech database,
Latin-American Spanish speech database.
The dataset consists in 2550 utterances adding up to a total of
130,000 phone instances, labeled
by expert phoneticians. Recordings are uttered by 206 native American English speakers.
Results showed that for a subset of the phones, the combination of supervectors and
dynamic features reduce the error compared with using supervectors only, thus supporting
the hypothesis that both set of features carry complementary information. Both feature-level
and score-level combinations gave similar results, and therefore it is not possible
to pick one type of combination as the best.

\bigskip

\noindent\textbf{Keywords:} Computer-Assisted Language Learning, Pronunciation Scoring, Phone Level, Support Vector Machines, Gaussian Mixture Models, Supervectors, Legendre Polynomials, Discrete Cosine Transform, Features Combination, Score Combination


