\chapter*{Abstract}
\addcontentsline{toc}{chapter}{\numberline{}Abstract}

\noindent
Technological advances of last decades have allowed the development and expansion
of Computer-Assisted language learning (CALL) systems. These systems assist
second language learners in different tasks regarding grammar, vocabulary and
pronunciation. In the current work, we focus on \textit{Pronunciation Assessment},
a particular subfield of pronunciation.
\textcolor{red}{
Pronunciation Assessment consists in
determining whether a recorded speech was correctly or incorrectly pronounced.
The analysis is performed at a predefined level, such as sentence, word or
phone level.
}

%assigning scores to recorded speech at a predefined
%level (such as sentence, word or phone level), according to how well or bad
%the speech segment was pronounced.

Currently, whenever performing pronunciation assessment, the most reliable estimates are
obtained from paragraphs or long sentences. On the other hand, the smaller the unit
(and therefore the smaller the amount of information in the speech segment),
the less precise is
the estimate of the assessment.
However, pronunciation assessment systems that operate at shorter levels, such
as phone level,
not only can point out specific errors produced by the students but also
can be used by children that still have difficulties in pronouncing long sentences.
For these reasons,
in the current work we will focus on phone-level pronunciation assessment methods.

The more standard methods in the literature for pronunciation assessment at
phone level usually involve using generative approaches
based on Gaussian Mixture Models.
Usually, for each phone two individual GMMs are trained: one using the correctly
pronounced instances of that phone and the other one using the incorrect instances.
A standard way to make the assessment is to compute the
Likelihood-Ratio between the two models.
In a previous work in the pronunciation assessment field at phone level, a discriminative
approach based on \textit{Support Vector Machines} (SVM) trained on special features
called \textit{supervectors} was explored, leading to slightly better results than
generative models such as \textit{Gaussian Mixture Models} (GMMs). Supervectors
are derived from adapted GMMs that are trained using all the available
instances for a given phone.

In the current work, we use as reference and baseline system the SVM model
trained on supervectors in order to explore new features in the
phone-level pronunciation assessment field.
Even though both GMMs and supervectors
summarize the low level acoustic information of the speech segment,
they don't provide information about the temporal dependencies of the features.
Because of that reason, in the current work we study
features that model explicitly the dynamics of the
acoustic features over time. In order to do so, each feature is modeled independently
by a parametric function, from which the dynamic features are extracted.
% In this work we use SVM models to explore dynamic features that model
% temporal dependencies in each instance.
Two different parameterization techiniques are studied:
\textit{Legendre Polynomials} and \textit{Discrete Cosine Transform} (DCT).
The objective is to analyse if the proposed dynamic features carry
complementary information to supervectors features.
% both supervectors and dynamic features carry
% complementary information, which can be used to improve future models.

We train and test the baseline and the proposed methods on a
% The dataset for the current work is obtained from a Latin-American Spanish speech database,
Latin-American Spanish speech database.
The dataset consists in 2550 utterances adding up to a total of
130,000 phone instances, labeled
by expert phoneticians. Recordings are uttered by 206 native American English speakers.
Results showed that for a subset of the phones, the combination of supervectors and
dynamic features reduce the error compared with using supervectors only, thus supporting
the hypothesis that both set of features carry complementary information.

\bigskip

\noindent\textbf{Keywords:} Computer-Assisted Language Learning, Pronunciation Assessment, Phone, Support Vector Machines, Gaussian Mixture Models, Supervectors, Legendre Polynomials, Discrete Cosine Transform
