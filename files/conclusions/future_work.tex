\color{red}
\subsection{Future Work}

While the current work was being written (after the experiments were carried out)
the Argentine children's speech database that was previously mentioned in section
\ref{section:motivation} was finished being collected. The database contains speech
from Argentine children between 6 and 12 years old, which are currently learning
english. As \textit{Computer-Assisted Language Learning} (CALL) systems
for kids should preferably be based on short speech units (phones or words) because
of the difficulty of children in pronouncing longer segments, the combined features
explored in the current thesis, which are phone-level features, could be helpful
in developing the CALL system. For this reason, it would be interesting to test
the combined features along with the SVM model against the new database, in search
of potential gains.

Due to the increasing number of highly effective DNN-based solutions that can be found in
many fields (including pronunciation assessment,
as it was described in the Previous Work section \ref{section:prev_work}),
it would be worthwile to explore DNN solutions to pronunciation assessment
at phone level. Below are listed some of the systems that would be interesting to be explored
along with their features:

\begin{itemize}
  \item Feed Forward DNN trained on supervectors or dynamic features (a set of features per phone utterance)
  \item Feed Forward DNN trained on the MFCCs of each frame plus context (a set of features per frame)
  \item Long Short-Term Memory (LSTM) DNN trained on the MFCCs of each frame (a set of features per frame)
  \item Feed Forward or LSTM trained on more generic features, such as the spectrum of frequencies of a spectrogram
\end{itemize}

Finally, a different aspect that may lead to improvements and therefore it is worth studying
is the usage of \textit{Detection Cost Function} (DCF) as the performance measure
to be used instead of the \textit{Equal Error Rate} (EER). DCT can assign different
weights to \textit{False Positive Rate} and \textit{False Negative Rate}, thus
prioritizing one over the other. In the context of pronunciation assessment in
CALL systems, labeling a correctly pronounced utterance as incorrect should be
penalized stronger than labeling a mispronounced utterance as correct, in order
to avoid discouraging the students.

\color{black}
