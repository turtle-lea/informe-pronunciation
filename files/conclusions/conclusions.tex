In the current thesis, we explored different Dynamic Features to be used as complementary features to
Supervectors when classifying phone instances using an SVM classifier. As in the previous work \cite{main},
Support Vector Machines stands as a suitable model for discriminating mispronounced and correctly
pronounced phone instances.

When tested in isolation, Dynamic Features turned out to be effective features in regard to
phone assessment, producing better results than a naive system.
The results for both
Legendre and DCT methods were very similar, yet DCT produced slightly better results in combination with
supervectors.

Experiments showed that among all the phones that were analysed in the current work, only a subset of them
benefits from the combination of supervectors and dynamic features. A proper strategy when following this
technique is then to combine the features using a phone-dependent approach.

% The results obtained in the development and hold-out sets suggest that in fact dynamic features carry
The obtained results suggest that in fact dynamic features carry
information complementary to supervectors' information, which leads to a reduction of the errors
made by the classifiers. For most of the phones for which the combination experiments gave statistically significant gains in the development set, the experiments in the hold-out set also yielded gains,
indicating that the approach generalizes well to unseen speakers.
% though in less extent. Moreover, the degradation for the remaining phones
% is not statistically significant.

The number of instances for each phone was insufficient in many cases. The classes for many
phones are highly unbalanced, which led to training models with high variance an
hindered the computation of reliable statistics for the errors.
Some possible consequences
of this, that can be observed in the statistical analysis carried out in the hold-out set,
could be the fact that bootstrapping
confidence intervals were very wide for some of the phones,
or that McNemar's test did not yield statistically significant results for many of the phones
for which the same test had given statistically significant results in the development set.
% Working with a dataset with more instances for each phone would have increased the robustness
% of the models and the reliability of the statistics.

Finally, it is not possible to determine if any of the combination approaches is better than the other based
on the obtained results. Both Features Combination and Score Combination approaches yielded
similar results with no clearer differences. Again, using a bigger dataset would have helped to pick out one
method over the other.
