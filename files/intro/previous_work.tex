\section{Previous work} \label{section:prev_work}

In this section we will be reviewing representative works in pronunciation assessment
at phone level, which is the topic of interest in this thesis.

One of the simplest ways of scoring a phone pronunciation found in the literature
is to use segment duration scores (\cite{pronunciation_scoring_instruction, pronunciation_scoring_phone_segments_instruction}), which are obtained through the following
procedure. First, a duration distribution is generated for each phone
using the native training data.
Then, phone durations in frames for nonnative utterances are obtained
and its value is normalized to compensate for rate of speech.
Finally, phone-segment-duration scores are computed as
the log-probability of the normalized duration, using the duration
distributions previously mentioned. Phone durations are obtained from Viterbi alignments.
In the same works, other two methods that use \textit{Hidden Markov Models} (HMMs) to obtain
spectral matches and compute pronunciation scores are tested.
Phonetic time alignments for
nonnative utterances are generated from an HMM-based speech recognition system trained
with native instances. In order to do that, the text pronounced by the student
must be known. This is achieved by eliciting speech in a
constrained way, such as reading a predefined text.
From these phonetic segmentations two
probabilistic measures based on HMMs are computed as scores: log-likelihood and
log-posterior probabilities. The underlying asumption is that the logarithm of the likelihood
or the posterior of the speech data, computed by the Viterbi
algorithm using the HMMs trained with speech from native
speakers is a good mesasure of the similarity between the students's
speech and native-sounding speech.

For each phone segment the log-likelihood score \^{l} is computed as:
\begin{equation}
\label{eq:logLikelihood}
\hat{l} = \frac{1}{d} \sum_{t=t_{0}}^{t_{0}+d-1} log \ p(y_{t}|q_{i})
\end{equation}
where $p(y_{t}|q_{i})$ is the likelihood at time $t$ with observation vector $y_{t}$
given the phone class $q_{i}$, $d$ is the duration in frames of the phone segment
and $t_{0}$ is the starting frame index of the phone segment.
The likelihood is computed through a forced recognition phase by using a known ortographic
transcription of the speech signal. Duration normalization is done to
eliminate the dependency of the pronunciation score on the duration of the phone.

Alternatively, log-posterior scores can be computed for each time $t$:

\begin{equation}
\label{eq:logPosterior1}
P(q_{i}|y_{t}) = \frac{p(y_{t}|q_{i})P(q_{i})}{\sum\limits_{j=1}^{M} p(y_{t}|q_{j})P(q_{j})}
\end{equation}
Likelihoods in both numerator and denominator are computed in the same way as in
Equation \ref{eq:logLikelihood}.
The sum over $j$ runs over a set of context-independent models for all phone classes. $P(q_{i})$
represents the prior probability of the phone class $q_{i}$.

Finally, the posterior score $\hat{\rho}$ for the phone segment is defined as:
\begin{equation}
\label{eq:logPosterior2}
\hat{\rho} = \frac{1}{d}\sum_{t=t_{0}}^{t_{0}+d-1} log \ P(q_{i}|y_{t})
\end{equation}

To test each method, a database of nonnative read speech is transcribed and scored for
pronunciation quality by expert human raters. Log-posterior probabilities achieves the
highest correlation with human ratings, outperforming log-likelihood and normalized duration
scores.

A very similar approach to log-posterior probabilities named \textit{Goodness of Pronunciation}
(GOP) was developed at the same time as log-posteriors and
is investigated in some works \cite{gop_1, gop_2, gop_3}. The quality of
pronunciation for any phone $p$ is defined to be the duration-normalized log of the posterior
probability that the speaker uttered phone $p$ given the corresponding vector of observations
$y=\{y_{t_{0}}, \dotsc y_{t_{0}+d-1} \}$

\begin{equation}
\label{eq:gopScore}
GOP(p) = \left| log \ \left(\frac{p(y|q)P(p))}{\sum_{q \in Q}p(y|q)P(q))}\right) \right| \Biggm/ d
\end{equation}

The likelihoods $p(y|q)$ are obtained from the recognizer as a sum of the likelihoods over all
observations in the phone.
The difference between GOP scores and the log-posterior score technique previously mentioned
is that in GOP, the likelihood for both numerator and every phone in the denominator is
computed at segment level as a sum of log-likelihoods per frame over the
duration of the phone $p$, while in Equation \ref{eq:logPosterior1} log-posterior
probabilities are computed at the frame level and averaged over the segment's length.

A different technique is explored in \cite{phonological_rules}, which
studies the relationship between phonetic mispronunciations that occur in a word
and the actual word rating. The research is based on a set of phonological rules
representing phonetic mispronunciations between L1 (native language of the students)
and L2 (the target language). In this case, the corpus is obtained from Chinese Learners
of English and all the speech data of the corpus is phonetically transcribed by trained linguists.
In addition, crowdsourcing is used to collect perceptual gradations of word-level
mispronunciations and the WorkerRank algorithm is used to conduct quality control to filter
crowdsourced data in terms of reliability. The gradation score for a phonological rule
is obtained as the average of the aggregated values of the words (rated via crowdsourcing)
containing that rule. On the other hand, gradation for a word is calculated as either
the maximum value among all phonological rules that are present in
that word or a linear combination
of them. Even though the work is focus on the use of phonological rules in word-level
mispronunciation gradation, the same approach can be used to evaluate mispronunciation
gradation at phone level. In order to derive the phonological rules from the mispronunciations,
the previous strategy requires a phonetically labeled nonnative database.
Taking that into account, another paper \cite{phonological_rules_2} works well as complement
to \cite{phonological_rules}, because it develops a speech recognition system for automatic
mispronunciation detection capable of transcribing the learner's input speech.

So far, all the aforementioned methods but the last one are based on confidence measures
obtained from Automatic Speech Recognition (ASR) systems.
Scores measure how closely the utterance of the speaker matches the recognizer's
acoustic model. Mismatches result in low confidence scores, which provide a profile of the
speakers' production errors. Nevertheless, the accuracy of the assessment based on these
confidence scores can be quite low \cite{landmark_svm}.
In addition, ASR systems based on HMMs are both time-consuming to train and extremely vulnerable to
acoustic interference and variation in speaking style, and the conventional methods for
enhancing ASR performance often require enormous amounts of data collection and annotation,
as well as extensive training on representative material.
For those reasons, other types of classifiers are explored in many works, specially after
the 1990s.

\textit{Support Vector Machines} (SVMs) are a preferred choice
due to their excellent generalization capability and suitability for 2-class classification
problems. Moreover, in contrast to the confidence models described above, which do not take into
account misspronounced data, SVMs are trained directly to discriminate the two classes of interest.
Of course, this poses a new requirement: both correctly and incorrectly-pronounced data should be
available for training.

In \cite{detection_mispronunciation_dutch_vowel}, SVMs are used to discriminate
between good and mispronounced vowels in Dutch.
Three different types of features for training the models are evaluated: log-posterior-based
scores obtained from HMMs, MFCCs and a set of
phonetic features (first three formants, pitch and intensity). Despite the existence of nonnative-speech
databases for Dutch language, they were considered too small for the purpose of the research.
For that reason, phonemic
substitutions that induce vocalic errors are simulated by artificially introducing them in the native corpus. It is worth mentioning that training
and testing the models with artificially generated data is not recommended
because it can lead to models that do not generalize well to real nonnative data.
Having said that, the results show that the best
results are obtained by using MFCCs as features of the SVM models, followed by confidence-measure-based
scores and finally phonetic
features, though substantial improvements can be obtained by combining them.

In a different work \cite{svm_space_models}, SVM is used as the classifier and the
log-likelihood ratios between all the acoustic models and the model corresponding to the given
phone are employed as features for the classifier. In other words, given the observation
$y$, the feature vector for the classification problem is computed as:

\begin{equation}
\label{eq:psm}
[LLR(y|q,q_{1}), \ LLR(y|q,q_{2}) \dotsc LLR(y|q, q_{Q})],
\end{equation}
where $q$ is the canonical label of the observation being pronounced, {$q_{1}, q_{2}
\dotsc q_{Q}$} represent the set of acoustic models of all the phones and $LLR$ is defined as:

$LLR(y|q,q_{i})=log \ p(y|q_{i}) - log \ p(y|q)$, where likelihood is computed at segment
level as in Equation \ref{eq:gopScore} for GOP scores. The reason for choosing these features is
that $q$ can be mispronounced as any other phone and the likelihood ratio is a powerful
method to detect this kind of mispronunciation. Classifiers trained with this kind of features, however,
can only effectively deal with a phone mispronounced as another phone. This kind of acoustic models is
less effective for partially changed mispronunciations, that are the most frequent mispronunciations
in practice. In order to detect all kinds of mispronunciation, a technique called
\textit{Pronunciation Space Models} (PSMs) is introduced. The idea is to represent pronunciation variations
of different proficiency levels. The traditional phone-based model $q$ is expanded to
\{$q_{1}, q_{2} \dotsc q_{K}$\} by means of an unsupervised algorithm that does not require
speech data labeled with proficiency level.
This set of models represents $K$ types of pronunciation variations ranging from perfect to
totally wrong pronunciations, where $K$ is a tunable parameter. To build the PSM models for a
given phone $q$, all the instances are sorted according to their
posterior probabilities obtained from the original acoustic model for $q$, and
then uniformly split into $K$ group of the same size. Finally, an individual new acoustic
model is trained with the observation from each group via maximum likelihood estimation.
The feature vector dimension of \ref{eq:psm} is therefore increased from $Q$ to $Q*K$.
Experiments are carried out on the
Mandarin mispronunciation detection task for native Chinese speakers with various dialect accents,
and data labeled as ``correct'' or ``mispronounced'' by experienced human annotators is used to train and test the models.
Results show that SVM based on log-likelihood ratio features outperforms GOP scores (used as baseline), and some
improvements are obtained when adding PSMs.

A last example of SVM-based resarch is found in \cite{landmark_svm, landmark_svm_2}.
A landmark-based SVM is introduced and compared with a confidence
scoring method over 10 phonemes where
L2 English learners, whose native language is Korean, make frequent errors.
Landmark theory relies on the fact that humans can perceive distinctive
features using only spectral features extracted from the time frame including and adjacent to
a landmark (sudden signal change). A particular SVM for each phoneme is trained. Vowel SVMs are
trained using one or more frames from the vowel center. Frames from both onset and offset
(prevocalic and postvocalic position) are selected and used in the training of consonant
SVMs. The confidence score method shows a similar performance to SVM, though by combining
the two scores, statistically significant improvements are achieved for almost all phonemes.

A different strategy for discrimination of Dutch velar fricative /x/ versus the velar plosive
/k/ is studied in \cite{lda_weigelt}. Linear Discriminant Analysis (LDA) over two different sets
of features are explored and compared with previous approaches: aforementioned GOP scores and
Weigelt algorithm. The latter is based on three simple measures that can be easily obtained:
log root-mean-square (rms) energy, the derivative of log rms energy (\textit{Rate of Rise} or
ROR) and zero-crossing rate. Weigelt algorithm discriminate plosives from fricatives by using
ROR values, based on the fact that the release of the burst of the plosives causes an abrupt
rise in amplitude therefore yielding higher values compared with fricatives.
On the other hand, LDA weights are assigned to each feature to
find the linear combination of features
which best separates the classes. Selecting the most relevant features turns out to give an
advantage compared to other classifiers. The two LDA methods yielded the best performing
scores followed by GOP and Weigelt.

% \textcolor{red}{
%   Finally, some pronunciation assessment approaches based on Deep Neural Networks (DNNs),
%   the most popular machine learning technique at this moment, have been developed for the
%   last few years. In \cite{metallinou_2014}, a DNN-HMM System for ASR is explored leading
%   to substantial gains compared with previous GMM-HMM-based approaches for ASR,
%   improving the quality of speaking proficiency features. In \cite{hu_1, hu_2}, a
%   DNN-HMM based approach is taken in order to improve the computation of GOP scores and
%   posterior probabilites of the different phones. Additionally, a Neural Network (NN)
%   based, logistic regression binary classifier is explored to analyse the
%   classification performance and generalization capabilities. Even though DNNs are a very
%   promising area with regard to pronunciation assessment, and where there is still a lot
%   of work to do, the main objective for the current thesis is to explore and test
%   new \textit{features} for pronunciation assessment at phone level. For this reason,
%   in the current work SVM is kept as the main classifier. However, exploring solutions
%   that involve DNNs for refining the acoustic features or reducing
%   the classification errors (to
%   name a few), integrates the list of things to do in future work.
% }

\textcolor{red}{
  Finally, some pronunciation assessment approaches based on Deep Neural Networks (DNNs),
  the most popular machine learning technique at this moment, have been developed for the
  last few years. In \cite{metallinou_2014}, the use of context-dependent DNN Hidden
  Markov models (or CD-DNN-HMMs) is studied in order to improve the speech recognition
  performance for a better assessment of children English language learners.
  The experimental results showed that the DNN-based recognition system approach
  achieved a significant error reduction compared to the GMM-HMM approach, leading
  to an improvement in the quality of the extracted features and final spoken English
  proficiency scores.
  In \cite{hu_1, hu_2}, a CD-DNN-HMM acoustic model approach for English language
  is also explored as an alternative to
  conventional GMM-HMM models.
  The acoustic models are then used to generate GOP scores
  at phone level and tested in a language learning corpus with manual grading
  of pronunciation quality. The GOP scores, generated from the DNN-based acoustic
  models, correlate better with human scores than those generated from the
  conventional GMM approaches.
  Even though DNNs are very promising
  solutions to pronunciation assessment and there is still a lot
  of approaches to be studied, the main focus of the current thesis is to explore
  new features for pronunciation assessment at phone level based
  on a well known technique, such as SVM.
}

Relying on labeled nonnative speech data usually leads to more flexible models
with better performance than those trained with native speech
when working on pronunciation assessment for specific combinations of L1 and L2. On the one hand,
the set of common pronunciation errors tend to be particular for a given \textless L1, L2\textgreater \ pair. On the other hand, models trained
with nonnative speech are better at capturing the subtle differences between the nonnative
speech realizations that are considered acceptable versus the nonnative speech realizations
that are considered mispronounced, in a similar way that an annotator would do.
However, labeled nonnative speech databases are rarely available. Annotation
of nonnative speech is an expensive and time-consuming task, and in some cases it is not feasible.
Models trained with annotated nonnative speech usually have better performance
than those trained with native data, though the cost of nonnative database
collection and annotation is very high.

To conclude this section, the papers that set the starting point for the current thesis will
be reviewed. They are the two latest works in one of the most important lines of investigation
in the field of pronunciation assessment at phone level.
Both use a Latin-American Spanish speech database that includes recordings from
native and nonnative speakers, where nonnative speech
is produced by speakers whose native language is American English.

The first of these works \cite{detection_phone_level_mispronunciation_learning} is
a continuation of \cite{pronunciation_scoring_phone_segments_instruction},
where log-likelihood and log-posterior scores were proposed as effective methods for
assessing pronunciation. In \cite{detection_phone_level_mispronunciation_learning},
the same log-posterior method (Equation \ref{eq:logPosterior1} and \ref{eq:logPosterior2})
is applied using models trained on native
data and is used as baseline and compared with a proposed technique
based on log-likelihood ratio (LLR).
To compute LLRs, the phonetically labeled nonnative
database is used to train two different \textit{Gaussian mixture models} (GMMs) for each phone
class: one model is trained with the ``correct'' native-like pronunciations of a phone, while the
other model is trained with the ``mispronounced'' or nonnative pronunciations of the same phone.
In the evaluation phase, for each phone segment $q_{i}$, a length-normalized log-likelihood ratio
$LLR(q_{i})$ score is computed for the phone segment by using the ``mispronounced'' and ``correct''
pronunciation models $\lambda_{M}$ and $\lambda_{C}$ respectively:
\begin{equation}
\label{eq:logLikelihoodRatio}
LLR(q_{i}) = \frac{1}{d}\sum_{t=t_{0}}^{t_{0}+d-1} [log \ p(y_{t}|q_{i}, \lambda_{M}) - log \ p(y_{t}|q_{i}, \lambda_{C})]
\end{equation}
The normalization by $d$ allows the definition of unique thresholds for each phone class,
independent of the length of the segments. A mispronunciation is detected when the LLR is above that threshold.
The LLR method performed better than the posterior-based method for almost all phone classes,
specially for those with the highest consistency across transcribers. These results are in
line with the fact that models trained on nonnative speech data have better results than
those trained on native speech, as it was previously mentioned.

Finally, the most recent work of the series \cite{main} extends the research
proposing two new methods that also explicitly model both mispronunciations and correct pronunciations
by nonnative speakers. The LLR method developed in
\cite{detection_phone_level_mispronunciation_learning} is used as baseline and compared with
the new modeling techniques.

The first proposed method also uses LLR based on GMMs trained
on nonnative data (Equation \ref{eq:logLikelihoodRatio}). The difference resides in the way these GMMs are obtained.
In the baseline approach, for each phone class two different GMMs are trained: one model is
trained with the ``correct'' native-like pronuncations of a phone while the other is trained
with the ``mispronounced'' or nonnative pronunciations of the same phone. In the new approach,
the models for each class for a given phone are obtained by adaptation. The model to which they
are adapted is trained using all the samples from a given phone, ignoring the class (``correct''
or ``mispronounced''). Bayesian adaptation \cite{ubm} is used to adapt this class-independent
GMM to all the ``correctly'' pronounced training examples for a phone and obtain the adapted
``correct'' model for that phone. An analogous process is followed to obtain the adapted
``mispronounced'' model. For these class-dependent GMMs both the means and the mixture weights
are adapted to the class-specific data.

The second proposed method follows a discriminative approach. It uses the class-independent
GMM of the previous experiment to create supervectors by adapting this GMM to each phone instance.
The supervector for a certain phone instance is obtained by adapting the means and the mixture
weights of the original GMM to the acoustic feature vector representing the phone.
These supervectors are then used as features to train a linear SVM classifier.
To evaluate test instances of a given phone, its supervector is calculated by adaptation of the class-independent
GMM to the phone feature vector frames. Finally, the distance of that supervector to the SVM
hyperplane is taken as the score for the phone.

The two methods presented in \cite{main} show better results than the mispronunciation
detection system based on LLR of independently trained GMMs developed in
\cite{detection_phone_level_mispronunciation_learning}, which at the same time
outperforms the standard method that uses phone log posteriors
as scores \cite{pronunciation_scoring_phone_segments_instruction}.
This is one of the most important lines of research in pronunciation assessment
at phone level, positioning the methods described in \cite{main} among the best
systems in the field. For that reason, both generative and discriminative models are
used as baseline and as a starting point in the work here presented. A more detailed
explanation of these systems will be provided in the next section.
