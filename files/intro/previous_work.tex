There are many works in the area of pronunciation assessment and describing all of them would
be a time consuming task. For that reason, only some of the works related to 
pronunciation assessment at phone level will be reviewed.

Study of pronunciation quality has been a topic of study since early works. An initial approach
(\cite{pronunciation_scoring_instruction}\cite{pronunciation_scoring_phone_segments_instruction}) 
is based on \textit{Hidden Markov Models} (HMMs) to obtain 
spectral matches and compute pronunciation scores. Phonetic time alignments for
non-native utterances are generated from an HMM-based speech recognition system trained
with native instances as first step. From these phonetic segmentations two 
probabilistic mesasures based on HMMs are computed as scores: log-likelihood and 
log-posterior probabilities.

For each phone segment the log-likelihood score \^{l} is computed as:
\begin{equation}
\hat{l} = \frac{1}{d} \sum_{t=t_{0}}^{t_{0}+d-1} log \ p(y_{t}|q_{i})
\end{equation}
where $p(y_{t}|q_{i})$ is the likelihood of the current frame with observation vector $y_{t}$
given the phone class $q_{i}$, $d$ is the duration in frames of the phone segment 
and $t_{0}$ is the starting frame index of the phone segment. Time normalization allows to 
eliminate the dependency of the pronunciation score on the duration of the phone.

Alternatively, in order to compute log-posterior scores, a frame-based posterior probability
$P(q_{i}|y_{t})$ is computed for each frame belonging to a segment.
\begin{equation}
P(q_{i}|y_{t}) = \frac{p(y_{t}|q_{i})P(q_{i})}{\sum\limits_{j=1}^{M} p(y_{t}|q_{j})P(q_{j})}
\end{equation}
The sum over $j$ runs over a set of context-independent models for all phone classes. $P(q_{i})$
represents the prior probability of the phone class $q_{i}$. Finally, the posterior score $\hat{\rho}$ for the phone segment is defined as:
\begin{equation}
\hat{\rho} = \frac{1}{d}\sum_{t=t_{0}}^{t_{0}+d-1} log \ P(q_{i}|y_{t})
\end{equation}

A second group of papers introduce L1-L2 strategies. In \cite{detection_mispronunciation_instruction}
speech is modeled with networks of phone HMMs where each phone in the network can optionally
be pronounced either natively or not. Native phone models are initialized using a subset of
the native training speech data. Nonnative phone models are initialized using the subset of 
the nonnative data that was scored low by the human raters. A similar strategy is presented
in \cite{detection_phone_level_mispronunciation_learning}. A phonetically labeled nonnative
database is used to train two different \textit{Gaussian mixture models} (GMMs) for each phone
class: one model is trained with the "correct" native-like pronunciations of a phone, while the
other model is trained with the "mispronounced" or nonnative pronunciations of the same phone.
In the evaluation phase, for each phone segment $q_{i}$, a length-normalized log-likelihood ratio
$LLR(q_{i})$ score is computed for the phone segment by using the "mispronounced" and "corret"
pronunciation models $\lambda_{M}$ and $\lambda_{C}$ respectively:
\begin{equation}
LLR(q_{i}) = \frac{1}{d}\sum_{t=t_{0}}^{t_{0}+d-1} [log \ p(y_{t}|q_{i}, \lambda_{M}) - log \ p(y_{t}|q_{i}, \lambda_{C})]
\end{equation}
The normalization by $d$ allows definition of unique thresholds for the LLR for each phone class, 
independent of the length of the segments.

A recent work \cite{main} explores another method that also uses log-likelihood ratio based on 
GMMs (equation 1.4), except that the models for each class ("correct" and 
"mispronounced") are obtained by adaptation. In the same work, a discriminative system based
on \textit{Support Vector Machines} (SVM) is developed producing good results. Both systems
previously described are used as baselines in the work here presented so a detailed explanation
will be provided in next sections.