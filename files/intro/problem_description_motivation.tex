% (REHACER UNA VEZ TERMINADAS LAS DEM\'AS SECCIONES DEL INFORME)

\section{Problem Description and Motivation} \label{section:motivation}

Technological improvements of last years allowed the development and expansion of Computer
Assisted Language Learning (CALL) systems. These tools are aimed to help students along the
process of second language acquisition. One of the points where the
students should focus is in pronunciation learning, that usually requires a one-to-one teacher
interaction. Automatic pronunciation assessment is very useful in this
context because it provides an alternative way of learning with a performance close to human judgement,
cheaper and typically available at any time and any place.

% Pronunciation is a general term that covers a number of different aspects and can be measured
% with different features. A lot of work has been done in the area, and a good summary of
% existing research to date followed by remaining challenges can be found in
% \cite{where_we_are_go}.

% Sometimes is difficult to determine whether or not a
% pronunciation error is being made because there is no clear definition of right or wrong
% pronunciation. Rather there exists an antire scale ranging from unintelligible speech to
% native-sounding speech. Taking that into account, pronunciation errors can be divided into
% phonemic and prosodic error types. Examples of severe errors of the first type are
% phoneme substitution, deletion or insertion. Errors on the prosodic side can be
% categorized in terms of stress, rythm and intonation. This work is focused in phonemic
% error detection.

% Whenever performing pronunciation assessment, the smaller the unit the higher the
% uncertainty of the assessment. Currently, the most reliable estimates of pronunciation
% are obtained from paragraphs composed of several sentences, so that an evaluation of the
% speaker's overall pronunciation proficiency can be done. However, in order for the system
% to provide valuable feedback of the particular error that is being made an analysis at a
% much shorter level, like phone, is required. By following this approach, the exact error within
% a word can be identified.

When working in automatic pronunciation assessment an important decision to be taken is
whether or not considering L1, the native language of the learner, along the process. These
systems have improved speech recognition accuracy because they are designed
taking into account common known errors between L1 and L2. For example, native American English
students that are learning Spanish have difficulties pronouncing the diphtong [eu]
(e.g. ``eufemismo'') or
/r/ after [l] (e.g. ``alrededor''), [n] (e.g. ``sonrisa'') and [s] (e.g. ``disruptivo''),
which should be thrilled in all cases.
On the other hand, native Spanish students
that are learning English may have trouble pronouncing
different kind of vowels that are present in English
but not in the former. This approach, however, requires a labeled nonnative speech database
which is an expensive and time-consuming task, and
it is not feasible in some cases.
A database like that
is available
for the current thesis.
It contains collected
speech from native american english speakers that are learning
spanish, and it is used to train and test the different models.


% so different experiments are carried out using L1-based models, which
% are trained on collected speech from native american english speakers that are learning
% spanish.

% Whenever performing pronunciation assessment, the smaller the unit the higher the
% uncertainty of the assessment. Currently, the most reliable estimates of pronunciation
% are obtained from paragraphs composed of several sentences, so that an evaluation of the
% speaker's overall pronunciation proficiency can be done. However, in order for the system
% to provide valuable feedback of the particular error that is being made an analysis at a
% much shorter level, like phone, is required. By following this approach, the exact error within
% a word can be identified.

In the area of pronunciation scoring, the smaller the unit to be scored, the higher the
uncertainty in the associated score \cite{pronunciation_scoring_phone_segments_instruction}.
Currently, the most reliable estimates are obtained from paragraphs composed of several
sentences that can be used to characterize the speaker's overall pronunciation
proficiency \cite{main}. However, specific problems can be pointed out
when scoring smaller units, which is of great value because it allows the students to focus
on specific aspects of their speech production.

% This thesis is part of a bigger project led by Dr. Luciana Ferrer, whose objective is
% the development of a CALL system for Argentine children that are learning english. The goal
% is to generate pronunciation scores for students utterances at phone level. This way,
% all children may enjoy the benefits of using the system, including those who aren't capable
% of pronouncing entire paragraphs or long sentences due to lack of skills. In addition it will
% help to detect specific errors in the students pronunciation, so they can focus in that
% particular points when practicing the language.

% Even though the database for the current work uses American English as L1 and Spanish as L2,
% the same implementation can be used to train the main system of the global project once
% collected the children utterances.
% \underline{In the current thesis, a system based on L1 language is used to assess
% pronunciation of english speakers learning spanish}

The current work is part of a bigger project lead by Dra. Ferrer, with the objective
of developing a CALL system for Argentine children that are learning English.
The project has 3 main stages: The collection and annotation of a speech database
of Argentine children, the development of an \textit{Automatic Speech Recognition} (ASR) system and the development of a pronunciation scoring system that works at phone or word
level. The fact that the system works with short speech segments is specially important in
the context of a CALL system for children, because of the difficulty of children in
pronouncing longer segments.

\textcolor{red}{
  In the current thesis, we based on L1 strategies to
  explore alternatives to existing techniques in the
  pronunciation assessment field at phone level. The explored
  methods could be helpful
  during the development of the CALL system for Argentine children that are learning English.
}
